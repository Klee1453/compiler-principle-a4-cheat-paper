% !TEX root = ./main.tex
% Liveness Analysis
% ======================================================
\par \noindent 称变量 \texttt{x} 在语句 \texttt{s} 中是活跃的,iff:
存在使用变量 \texttt{x} 的语句 \texttt{s'} 且
有一条从 \texttt{s} 到 \texttt{s'} 的路径,该路径中没有对 \texttt{x} 的赋值/定义。
\par \noindent 称变量 \texttt{x} 在边 E 上是活跃的,iff:$x \in \text{use}(s')$ 且
有一条经过 E,到 \texttt{s'} 的路径,并且该路径中的所有节点的 $\text{def}()$ 均不包括 \texttt{x}。

\par \noindent 判断变量在某个位置是否活跃:不可判定。保守的近似算法:
\vspace{-10pt}
$$
\begin{array}{rl}
    \text{in}(n)  &= \text{use}(n) \cup (\text{out}(n) - \text{def}(n)) \\
    \text{out}(n) &= \bigcup_{s \in \text{succ}(n)} \text{in}(s)
\end{array}
$$
\vspace{-15pt}
\par \noindent 初始条件:$\text{in}(n) = \varnothing$,$\text{out}(n) = \varnothing$;迭代直到不动点。
一定收敛的原因:单调有界定理。
求并集: $\mathcal{O}(N)$,每一次迭代:$\mathcal{O}(N^2)$,至多迭代:$\mathcal{O}(2N^2)$;总复杂度 $\mathcal{O}(N^4)$。
如果选择合适的顺序,实际运行时间介于 $\mathcal{O}(N)$ 与 $\mathcal{O}(N^2)$。

\par \noindent 优化:倒序,从后往前,对每个节点先计算 out 再计算 in。

\par \noindent 静态活跃性:存在一条控制流路径;动态活跃性:运行时。动态活跃 $\Rightarrow$ 静态活跃。
静态活跃性是一种保守的近似估计,它认为所有分支都会被执行。上面的算法是一种静态活跃性分析,这体现在第二个式子中。

% !TEX root = ./main.tex
% Abstract Syntax
% ======================================================
\par \noindent 属性文法 = 上下文无关文法 + 属性 + 属性计算规则。
属性: 描述文法符号的语义特征,如变量的类型、值等(例: 非终结符 E 的属性 E.val = 表达式的值);
属性计算规则(语义规则): 与产生式相关联、反映文法符号属性之间关系的规则(例: 如何计算E.val),
仅表明属性间抽象关系,不涉及具体实现细节,如计算次序等。
应用:程序分析表达式的类型、值、执行代价;抽象语法树生成、中间代码甚至汇编生成。
可通过 Parser 生成器支持的语义动作(Semanticaction)实现计算,并应用于抽象语法树生成等场景。

\par \noindent 抽象语法树:不依赖于具体语法细节的树形结构,比解析树简洁,没有无意义的终结符节点。

\par \noindent 关于错误位置:parser 维护一个位置堆栈以及语义值堆栈,使每个符号的位置可供语义操作使用(Bison);
定义一个非终结符 \texttt{pos},其语义值是源位置(行号,或行号和行内位置)(Yacc)。
% !TEX root = ./main.tex
% Semantic Analysis
% ======================================================
\par \noindent 语义分析:引用的维度是否与声明匹配、数组访问是否越界、变量应该存储在哪里\dots(这些问题取决于值而不是语法)
\par \noindent 狭义语义分析:通过 AST 确定程序的一些静态属性,例如名称的范围和可见性(每个变量在使用前都已声明)、
变量、函数和表达式的类型(每个表达式都有适当的类型、函数调用符合定义)。

\par \noindent 符号表:环境的实现;环境:绑定的集合;绑定:$\text{Name/Symbol} \mapsto \text{Meaning/Attribute}$;属性:类型、值、函数签名等。
环境:$\sigma_1 = \sigma_0 + \{a \mapsto \text{int}, b \mapsto \text{int}, c \mapsto \text{int}\}$,
$\sigma_2 = \sigma_1 + \{a \mapsto \text{str}\}$,等式右侧的符号表将覆盖左侧的符号表(现在 $a \mapsto \text{str}$)。

\par \noindent 符号表的实现:Imperative(例:bucket list / hash table)使用哈希查找对应标识符在表中的位置(bucket),
插入新环境时,在对应标识符 bucket 顶部插入新绑定,恢复环境时,从对应 bucket pop 绑定。
Functional(例:persistent BST)使用可持久化维护数据结构的历史,退出 scope 时回到相应的历史状态,
插入新环境时只复制标识符在树中的所有祖先(避免完整拷贝所有旧版本)。